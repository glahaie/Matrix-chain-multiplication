%dev_3.tex
%Par Guillaume Lahaie
%LAHG04077707
%
%%%%%%%%%%%%%%%%%%%%%%%%%%%%%%%%%%%%%%%%%
% Simple Sectioned Essay Template
% LaTeX Template
%
% This template has been downloaded from:
% http://www.latextemplates.com
%
% Note:
% The \lipsum[#] commands throughout this template generate dummy text
% to fill the template out. These commands should all be removed when 
% writing essay content.
%
%%%%%%%%%%%%%%%%%%%%%%%%%%%%%%%%%%%%%%%%%

%----------------------------------------------------------------------------------------
%	PACKAGES AND OTHER DOCUMENT CONFIGURATIONS
%----------------------------------------------------------------------------------------

\documentclass[10.9pt]{article} % Default font size is 12pt, it can be changed here
\renewcommand{\familydefault}{\rmdefault}
\renewcommand{\thesubsection}{\alph{subsection}}

%Pour l'encodage avec accents
\usepackage[utf8]{inputenc}
\usepackage{longtable}
\usepackage{algorithm2e}

%\usepackage{helvet}
%\renewcommand{\familydefault}{\sfdefault}

%Pour INF4100 - devoir 3
\usepackage{tikz}
\usepackage{algorithm2e}

\usepackage{afterpage}
\usepackage{appendix}
\usepackage{graphicx} % Required for including pictures
\usepackage{listings}
\usepackage{mathtools}

\usepackage[left=2.2cm,top=2.2cm,right=2.2cm,bottom=2.2cm,nohead]{geometry} % Required to change the page size to A4
\geometry{letterpaper} % Set the page size to be A4 as opposed to the default US Letter

\usepackage{float} % Allows putting an [H] in \begin{figure} to specify the exact location of the figure
\linespread{1.2} % Line spacing

%\setlength\parindent{0pt} % Uncomment to remove all indentation from paragraphs

\graphicspath{{./Pictures/}} % Specifies the directory where pictures are stored
\usepackage[french,english]{babel}

%Comportement d'un paragraphe
\setlength{\parskip}{\baselineskip}%
\setlength{\parindent}{0pt}%

%Widows/orphans
\widowpenalty10000
\clubpenalty10000

\usepackage[hidelinks]{hyperref}

%Meta-info
\title{INF4100 - devoir 3}
\author{Guillaume Lahaie}
\date{Remise: 1er avril 2014}

\hypersetup{
  pdftitle={INF4100 - devoir 3},
  pdfauthor={Guillaume Lahaie}
}

\newcommand\blankpage{%
  \null
  \thispagestyle{empty}%
  \addtocounter{page}{-1}%
  \newpage}

\begin{document}
\selectlanguage{french}
\fussy

%----------------------------------------------------------------------------------------
%	TITLE PAGE
%----------------------------------------------------------------------------------------

\begin{titlepage}

\newcommand{\HRule}{\rule{\linewidth}{0.5mm}} % Defines a new command for the horizontal lines, change thickness here

\center % Center everything on the page

\textsc{\LARGE Université du Québec à Montréal}\\[1.5cm] % Name of your university/college
\textsc{\Large INF4100}\\[0.5cm] % Major heading such as course name

\HRule \\[1.5cm]
{ \huge \bfseries Devoir 3}\\[0.4cm] % Title of your document
\HRule \\[1.5cm]

\begin{minipage}{0.4\textwidth}
\begin{flushleft} \large
\emph{Par:}\\
Guillaume Lahaie \\ LAHG04077707 % Your name
\end{flushleft}
\end{minipage}
~
\begin{minipage}{0.4\textwidth}
\begin{flushright} \large
\emph{Remis à:} \\
Louise Laforest % Supervisor's Name
\end{flushright}
\end{minipage}\\[4cm]

{\large \emph{Date de remise:} \\ Le 1$^{er}$ avril 2014}\\[3cm] % Date, change the \today to a set date if you want to be precise

%\includegraphics{Logo}\\[1cm] % Include a department/university logo - this will require the graphicx package

\vfill % Fill the rest of the page with whitespace

\end{titlepage}
\blankpage

%----------------------------------------------------------------------------------------
%	TABLE OF CONTENTS
%----------------------------------------------------------------------------------------

\tableofcontents % Include a table of contents

\newpage % Begins the essay on a new page instead of on the same page as the table of contents 

%----------------------------------------------------------------------------------------
% SECTIONS DU DOCUMENT
%----------------------------------------------------------------------------------------


\section{Numéro 1.}

\subsection{Comparez les temps d'exécution des trois algorithmes}

Voici les temps d'exécution présentés sous forme graphique. Le premier
graphique présente les résultats pour des fichiers de chaines de 5 à 20 
chaines de matrices, et le second des chaines de 21 à 100 matrices.

Le premier graphique démontre que pour le temps d'exécution s'accroit
de façon exponentiel, et donc après 20 matrices, le temps d'exécution
devient trop long pour représenter sur le graphique.

On peut voir ensuite que le temps d'exécution pour l'algorithme diviser
pour régner avec stockage et l'algorithme en programmation dynamique
croit à un rythme similaire, mais l'algorithme de programmation
dynamique, en évitant les appels récursifs, est plus efficace.


\subsection{Algorithme qui affiche l'expression de parenthésage}

Voici un algorithme, ayant comme source le livre ...

\begin{algorithm}
 \SetKwInput{Donnees}{donnees}
 \SetKwInput{Antecedents}{antécédents}
 \SetKwInput{Consequents}{conséquents}
 \Donnees{ \emph{frontiere}: matrice n*n indicé de 1 à $n$ \\ 
	   \emph{i}: un indice entre 1 et $n$ \\
	   \emph{j}: un indice entre 1 et $n$}
 \Consequents{Le parenthésage optimal est affiché à l'écran}
 \Deb{
    \Si{$ i == j$}{
	imprimer("A"+$i$)
	}
      \Sinon{
	imprimer("(")\\
        imprimerParenthesage($frontiere$, $i$, $frontiere[i][j]$)\\
        imprimerParenthesage($frontiere$, $frontiere[i][j]+1$, $j$)\\
        imprimer(")")\\
        }
    }       
\end{algorithm}
\end{document}
